% ------------------------------------------------------------------------
% file `6GT_phys2_ondes-exercise-6-exercise-body.tex'
%   in folder `xsim/'
%
%     exercise of type `exercise' with id `6'
%
% generated by the `exercise' environment of the
%   `xsim' package v0.21 (2022/02/12)
% from source `6GT_phys2_ondes' on 2024/04/30 on line 54
% ------------------------------------------------------------------------
    Un objet d'une masse de 500[g] est attaché à un ressort. Le graphique ci-dessous représente son élongation au cours du temps.
    \begin{enumerate}[a)]
        \item Calcule la constante de raideur du ressort.
        \item Calcule l'intensité de la force résultante agissant sur l'objet lorsque l'élongation est maximale.
    \end{enumerate}
    \begin{figure}[ht!]
        \centering
        \begin{tikzpicture}[scale=0.75]
            \tikzset{>=latex}
            \tkzInit[xmin=0,xmax=1,ymin=-10,ymax=10,xstep=0.1,ystep=2]
            \tkzGrid
            \tkzDrawX[label={$t [s]$},below left=25pt]
            \tkzDrawY[label={$Y [mm]$},right=5pt]
            \tkzAxeXY[label={}] %This macro combines the four macros: \tkzDrawX\tkzDrawY \tkzLabelX\tkzLabelYnode font=\tiny]
            \tkzFct[domain=0:1,red]{8*sin(0.5*pi*x)}
        \end{tikzpicture}
    \end{figure}
