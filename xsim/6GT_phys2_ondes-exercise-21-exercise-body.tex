% ------------------------------------------------------------------------
% file `6GT_phys2_ondes-exercise-21-exercise-body.tex'
%   in folder `xsim/'
%
%     exercise of type `exercise' with id `21'
%
% generated by the `exercise' environment of the
%   `xsim' package v0.21 (2022/02/12)
% from source `6GT_phys2_ondes' on 2024/04/30 on line 138
% ------------------------------------------------------------------------
    On attache une corde horizontale d'une masse linéique de \(4,2 \times 10^{-4} [kg \cdot m^{-1}]\) par l'une de ses extrémités à un vibrateur oscillant à 60[Hz] avec une faible amplitude. La corde passe par une poulie située à une distance \(L=1,4[m]\) de l'extrémité fixée au vibrateur et des poids sont suspendus à l'autre extrémité.
    Quelle doit être la masse suspendue à l'extrémité de la corde pour produire 5 ventres ? On considère que l'extrémité fixée au vibrateur correspond à un noeud, car l'amplitude du signal est faible par rapport à celle des interférences.
