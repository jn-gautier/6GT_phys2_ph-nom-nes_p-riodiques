% ------------------------------------------------------------------------
% file `6GT_phys2_ondes-exercise-18-solution-body.tex'
%   in folder `xsim/'
%
%     solution of type `exercise' with id `18'
%
% generated by the `solution' environment of the
%   `xsim' package v0.21 (2022/02/12)
% from source `6GT_phys2_ondes' on 2024/05/31 on line 96
% ------------------------------------------------------------------------
    Dans cette situation, \(\mu\) est identique pour les deux instruments puisqu'il s'agit de la même corde. Donc :
    \(\frac{F_{T1}}{L_1 ^2}=\frac{F_{T2}}{L_2 ^2}\) \\
    Si on remplace \(L_1\) et \(L_2\) par leurs valeurs respectives, on obtient :\\
    \(F_{T1}=0,3906 \cdot F_{T2}\) .\\
    \(F_{T1}\) est donc plus petit que \(F_{T2}\), la deuxième corde est plus tendue que la première. Le rapport entre les deux tensions est de \(0,3906\).
